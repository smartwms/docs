\chapter{Problema}
\textual
O problema consiste em implementar soluções para a localização de ativos em um espaço físico de um armazém.
O problema de localização outdoor para espaços sem muitos obstáculos, já tem como padrão a tecnologia de GPS, entretanto, quando se trata de espaços outdoor com muitos obstáculos ou espaços indoor, diversas tecnologias são levantadas com aspectos positivos e negativos \cite{art4}, porém a escolha de uma em relação a outra depende do problema em questão

Os principais parâmetros para um sistema ideal de localização, incluem o sistema estar disponivel atualmente em dispositivos de usuários, ser econômico, ser energeticamente eficiente, possuir uma ampla faixa de recepção, alta precisão de localização, baixa latência e alta escalabilidade \cite{art2}.

O problema é que cada tecnologia em conjunto com a aplicação a que se destina pode não apresentar uma ou mais das caracteristicas ideais desejáveis. Assim, nesse trabalho, será encontrado a solução que melhor se adeque aos parâmetros de um sistema de localização ideal para o armazém especifico da Sansung em que o trabalho será feito.

Além disso, o sistema proposto deverá ser facilmente integrável com o \textit{warehouse management system} (WMS) em uso no armazém em questão, de forma a minimizar mudanças na infraestrutura do armazém. Grandes mudanças implicam em custos adicionais, o que é um grande problema em termos comerciais.
