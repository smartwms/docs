\chapter{Introdução}
\textual
A função primaria de um armazém é receber mercadorias de uma fonte, guardar até ser requerido, e enviar para o usuário apropriado quando exigido \cite{tompkins}. É uma parte fundamental da cadeia de suprimentos de qualquer rede de mercadorias.

Nos últimos anos, os armazéns vêm cada vez mais enfrentando maiores demandas por custo e produtividade. Estão se tornando uma parte vital para muitas empresas, porém sua complexidade também está aumentando. Consequentemente, o planejamento e controle dos processos de um armazém, também conhecido como \textit{warehouse management}, tem se tornado uma tarefa desafiadora \cite{faber}.

Nesse contexto, diversas soluções estão sendo buscadas para abaixar os custos, aumentar a produtividade, e melhorar o planejamento dos processos.
Um ponto de melhoria, é a localização de ativos em um armazém. Com dados precisos da localização de todos os ativos em tempo real é possível utilizar algoritmos que visam melhorar a eficiência do armazém.
Os ambientes de armazém, podem ser os mais variados, desde armazéns indoor em ambientes mais controlados, até armazéns outdoor em que espaços são compartilhados entre várias empresas.

Para a realização da localização de produtos em um local indoor e outdoor, diversas tecnologias, técnicas e algoritmos podem ser utilizados, cada um com seus aspectos positivos e negativos, e a escolha de um em detrimento de outro pode levar em conta diversos aspectos. As principais tecnologias utilizadas para a localização são: infravermelho (IR), ultra-som, som audível, sensores magnéticos, sensores óptico, radiofrequência (RF) e luz visível. As tecnologias de RF incluem Bluetooth, banda ultralarga (UWB), wireless sensor network (WSN), rede Local sem fio (WLAN), identificação por radiofrequência (RFID), Near Field Communication (NFC), WiFi, entre outros.
Ja as técnicas de estimativa de localização incluem: Angle of Arrival (AOA), Time of Arrival (TOA), Time Difference of Arrival (TDOA) and Received Signal Strength Indication (RSSI).  \cite{art1,art2}

Essa pesquisa é motivada pela \textit{Samsung SDS Cello Logistics} para a localização de seus ativos em um armazém localizado no Porto de Tubarões no Espirito Santo. Dada a conjuntura do espaço físico de tal porto, somado com os baixos consumos energéticos do Bluetooth, seus baixos custos, e sua alta portabilidade e facilidade de desenvolvimento, tal tecnologia foi escolhida para o desenvolvimento do trabalho, como será demonstrado nas próximas partes.
