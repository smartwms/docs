\section{Necessidade}
\textual
À medida que mais empresas buscam cortar custos e melhorar a produtividade dentro de seus armazéns e centros de distribuição, a etapa de coleta tem se tornado um caso de estudo cada vez mais detalhado. \textit{Order Picking} (Coleta de pedidos) - o processo de coleta de produtos do estoque em resposta ao pedido de um cliente - é a operação mais trabalhosa em armazéns com sistemas manuais e uma operação muito intensiva em armazéns com sistemas automatizados. Por essas razões, os profissionais de armazenamento consideram a coleta de pedidos como a área de prioridade máxima para melhorias de produtividade \cite{art3}.
Nesse contexto de busca de melhorias para a coleta e consequente aumento de produtividade, se encaixam as necessidades do estudo em questão de criar métodos melhores de localização de produtos em um armazém, de forma a otimizar a coleta e estoque de produtos.

No sistema atual, de estoque e coleta do espaço da Samsung, códigos de barra em localidades fixas, são utilizados para a demarcação da localização em que um produto foi estocado. Entretanto, tal método não está se mostrando efetivo em um porto outdoor, com espaços para armazenamento que podem mudar a cada chegada de um novo contêiner e que estão sujeitos à adversidades climáticas. Tal situação exigiu a busca de novas soluções para a localização de produtos a fim de manter padrões elevados de produtividade.
